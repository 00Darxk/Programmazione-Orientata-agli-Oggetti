\documentclass{article}

\usepackage{cancel}
%\usepackage{tikz}
\usepackage{amsmath}
\usepackage[includehead,nomarginpar]{geometry}
\usepackage{graphicx}
\usepackage{amsfonts} 
\usepackage{verbatim}
\usepackage{mathrsfs}  
\usepackage{lmodern}
\usepackage{braket}
\usepackage{bookmark}
\usepackage{fancyhdr}
\usepackage{romanbarpagenumber}
\usepackage{minted}
%\usepackage{subfig}
\usepackage[italian]{babel}
%\usepackage{float}
\allowdisplaybreaks

\setlength{\headheight}{12.0pt}
\addtolength{\topmargin}{-12.0pt}

\hypersetup{
    colorlinks=true,
    linkcolor=black,
}

\makeatother

\numberwithin{equation}{subsection}

\fancypagestyle{link}{\fancyhf{}\renewcommand{\headrulewidth}{0pt}\fancyfoot[C]{Sorgente del file LaTeX disponibile al seguente link: \url{https://github.com/00Darxk/Programmazione-Orientata-agli-Oggetti}}}

\begin{document}

\title{%
    \textbf{Programmazione Orientata agli Oggetti}  \\ 
    \large Appunti delle Lezioni di Programmazione Orientata agli Oggetti \\
    \textit{Anno Accademico: 2023/24}}
\author{\textit{Giacomo Sturm}}
\date{\textit{Dipartimento di Ingegneria Civile, Informatica e delle Tecnologie Aeronautiche \\
Università degli Studi ``Roma Tre"}}

\maketitle
\thispagestyle{link}

\clearpage


\pagestyle{fancy}
\fancyhead{}\fancyfoot{}
\fancyhead[C]{\textit{Programmazione Orientata agli Oggetti - Università degli Studi ``Roma Tre"}}
\fancyfoot[C]{\thepage}
\pagenumbering{Roman}

\tableofcontents

\clearpage

\pagenumbering{arabic}



%% TODO: fix order of contents / merge (sub)section



\section{Introduzione al Linguaggio Java}

% FdI, ASD 
%% valutare la qualità del codice !!!
%% programma ppt. README 
% "testi di riferimento" ppt.

%% ppt. strumenti da installare 
% diagramma di flusso del corso sui ppt. 

%% storia
Java venne introdotto nel 1995, progettato per programmare dispositivi embedded, ma velocemente si utilizzò per scopi diversi da quelli descritti dai suoi 
creatori. Il team di Java sviluppò un sistema di embedded per i browser, le applet, delle piccole applicazioni che potevano essere eseguite all'interno 
del browser. Questo fu uno dei primi motivi per il successo di Java, un'altro aspetto importante per la diffusione di Java fu la sua caratteristica di 
semplificare aspetti di C e C++. Utilizzando un linguaggio più semplice diminuiva il costo associato ad un programmatore, per cui l'adozione di Java 
venne trainata dalle aziende alla ricerca di abbassare gli stipendi dei suoi dipendenti. In seguito venne utilizzato per creare la macchina virtuale su 
cui operano tutti i dispositivi Android. 

Java è stato ideato con la capacità di mantenere la retro-compatibilità, nonostante le spinte di innovazione tecnologica che il linguaggio ha sostenuto 
negli anni. Nonostante la sua età Java è uno dei linguaggi di programmazione più usati dalle aziende. 
La diffusione più importante di Java si ebbe tra il '95 ed il '98. %% ppt. 


Java fu il primo progetto industriale che introdusse un nuovo concetto di portabilità su scala industriale, un processore virtuale, poiché non viene progettato per un 
calcolatore fisico, ma la piattaforma Java Virtual Machine (JVM), per cui uno stesso programma Java può essere eseguito su tutte le piattaforme senza dover ricompilare 
nuovamente il codice per quella specifica piattaforma. Questa rappresenta una rivoluzione in ambito industriale, il codice oggetto Java ``bytecode'' prodotto dalla 
compilazione, come un file di estensione \verb|.class|. Questo bytecode può quindi essere eseguito da qualsiasi piattaforma che dispone di un'implementazione della 
JVM. Invece programmi creati da un linguaggio più vecchio come C, producono un codice oggetto che presenta istruzioni macchina uniche alla piattaforma in cui è stato 
compilato. Astraendo il processore fisico si è reso possibile la creazione di programmi che non devono essere ricompilati per ogni piattaforma su cui deve essere eseguito. 

Gli applet realizzati tramite Java, venivano eseguiti dalla JVM, e permise la creazione di primi siti web, rivoluzionando il paradigma di programmazione dell'epoca. 
La sintassi e la semantica del linguaggio sono descritte in un documento noto come Java Language Specification, per risolvere eventuali ambiguità. Inoltre fu uno dei primi 
linguaggi ad aver introdotto librerie incluse insieme al compilatore, sulla piattaforma Java. 

\subsection{Rapporto con il Linguaggio C}

Il linguaggio C viene descritto come un linguaggio ad un medio livello di programmazione, di alto livello rispetto all'assembly, e permette di realizzare programmi ad alte 
prestazioni, utilizzato per scrivere sistemi operativi. Java invece venne introdotto in un ambiente diverso rispetto al linguaggio C, per cui si indica come un linguaggio 
di alto livello di astrazione, non è un linguaggio che cerca le migliori prestazioni, ma cerca di creare programmi semplici e portabili. Nel periodo in cui venne introdotto 
Java l'andamento delle prestazioni dei calcolatori permise di realizzare linguaggi che non avessero come scopo principale il risparmiare risorse del sistema. Il linguaggio 
C++ permette di avere prestazioni simili al linguaggio C, ma introducendo paradigmi realizzati in Java. Ma si preferisce per realizzare programmi ad alte prestazioni il 
C, rispetto al C++, poiché fornisce troppe scelte, complicando l'implementazione e l'uso del linguaggio. 



In Java una dichiarazione di una variabile comprende la sua inizializzazione ad un valore nullo. Le dichiarazioni e assegnazioni in Java si ottengono mediante la stessa 
sintassi del linguaggio C. Si introduce un tipo \verb|boolean| utilizzato per rappresentare i due possibili valori booleani \verb|true| e \verb|false|. Le descrizioni di 
tutti i tipi utilizzati in Java vengono definite nel JLS. Entrambi sono due linguaggi staticamente tipati, ovvero a tempo di compilazione deve essere noto il tipo di ogni 
dato utilizzato nel programma. 
Java introduce il tipo \verb|String| utilizzato per rappresentare stringhe, non è un dato primitivo, e permette l'uso molto più semplice delle stringhe rispetto a C. Rappresenta 
un tipo particolare di classe, in appoggio al compilatore per favorire la loro gestione da parte del compilatore. Le stringhe vengono inizializzate a letterali stringhe, 
contenuti tra due doppi apici \verb|"  "|. Da Java 13 è possibile scrivere stringhe letterali multi-linea, utilizzando tripli doppi apici \verb|"""   """|. 

La sintassi di controlli di flusso è esattamente la stessa del linguaggio C, ma le condizioni sono diverse a causa del tipo booleano. 

Una grande differenza tra i due linguaggi consiste nella diversa diagnostica a tempo di compilazione e di esecuzione degli errori, poiché in Java i gli errori ed i messaggi 
di errore forniscono informazioni molto utili per la sua risoluzione. 
Gli errori a tempo di compilazione sono altrettanto efficaci in Java quanto in C, ma  la differenza principale consiste negli errori a tempo di esecuzione tra i due linguaggi, 
poiché in C, in molte piattaforme, viene fornito un errore estremamente generico, in base alla versione di C, del compilatore, e da altre condizioni. 

\subsection{Paradigma Orientato agli Oggetti}

Il paradigma procedurale, usato nel linguaggio C, separa in maniera netta tra il codice e le operazioni, ovvero la memoria e ed il processore, seguendo l'architettura 
definita da von Neumann. Anche se rappresenta il modo più naturale per scrivere programmi, nell'industria è stato ampiamente superato a favore del paradigma orientato agli 
oggetti. 
Questo paradigma segue la filosofia secondo cui le operazioni e lo stato sono connesse tra di loro, accomunate utilizzando ``oggetti''. Si è quindi rimossa la divisione imposta 
sui vecchi linguaggi di programmazione, che seguono l'architettura reale di un calcolatore. Per cui un problema è più facilmente modellabile se viene diviso in una pluralità 
di oggetti e classi di oggetti, contenenti lo stato e le operazione eseguibili su quell'oggetto. Questi oggetti si scambiano informazioni tra di loro, invocando particolari 
comandi; in questo modo gli oggetti conoscono ed interagiscono con altri esemplari, anche dello stesso tipo, o classe. Per conoscersi, questi oggetti contengono riferimenti 
agli altri oggetti. 

Questo modo di interpretare un problema è vicino al nostro modo di pensare, il che aiuta nella creazione di programmi, utilizzando linguaggi di programmazione che seguono 
questo paradigma. Utilizzando un linguaggio come C, progettato per il paradigma procedurale, è possibile programmare seguendo il paradigma orientato agli oggetti, ma non essendo 
stato creato con questo scopo, la realizzazione di programmi è notevolmente più complesso. 


L'esecuzione di un programma è uno scambio di messaggi tra questi oggetti connessi tra di loro tramite riferimenti interni, creando una rete di oggetti, modificando lo stato 
degli oggetti coinvolti nelle operazioni. Si può rappresentare la rete di oggetti tramite un diagramma di oggetti, indicati come rettangoli contenenti diversi campi, ed 
eventuali riferimenti ad altri oggetti, rappresentati come archi orientati verso altri oggetti. 

Ogni oggetto contiene lo stato, un comportamento, le operazioni che offre, ed un identità, per differenziare diversi esemplari della stessa classe di oggetti. Si è dimostrato 
nel tempo come questo paradigma permette di scrivere e mantenere il codice molto più efficiente rispetto ad altri paradigmi. 
Gli oggetti vengono definiti tramite classi, rappresentate come file diversi del programma, contenenti lo stato, e la definizione dei metodi. Ogni oggetto creato da una classe 
possiede un'identità unica, e presenta lo stesso stato e le operazioni della classe di appartenenza. Le classi sono quindi delle istruzioni di montaggio, o fabbriche che 
producono oggetti. 


Per studiare questo paradigma, si utilizzeranno esempi di forme geometriche, come studio di caso più semplice e limitato, ed uno studio di caso più esteso, mirato al 
refactoring, chiamato ``diadia''. 


In Java ogni file contenente classi deve essere salvato con lo stesso nome della classe contenente, con estensione ``.java''. Le operazioni di oggetti in Java si indicano 
come metodi, invece di funzioni come nel linguaggio C, poiché rappresentano operazioni nel contesto di una classe, mentre le funzioni possono operare al di fuori di una classe. 

\subsection{Oggetti e Riferimenti}

Si considera un esempio semplice di una classe chiamata \verb|Punto|, che rappresenta un punto in un piano di coordinate cartesiane bidimensionali:
\begin{minted}{java}
public class Punto{
    // stato dell'oggetto
    private int x;
    private int y;

    // metodi propri della classe
    public void setX(int posX){
        this.x = posX;
    }
    public void setY(int posY){
        this.y = posY;
    }
    public int getX(){
        return x;
    }
    public int getY(){
        return y;
    }
}
\end{minted}
Le variabili della classe \verb|Punto| \verb|x| e \verb|y|, vengono chiamate variabili d'istanza. La creazione di un oggetto di una classe si ottiene tramite l'operatore \verb|new|, 
secondo la seguente sintassi:
\begin{minted}{java}
// creazione ed assegnazione di un nuovo oggetto Punto tramite riferimento
Punto origine = new Punto();
\end{minted}
Questa riga di codice invoca un costruttore tramite l'operatore \verb|new|, restituisce un riferimento ad un oggetto del tipo \verb|Punto| appena creato, e viene conservato 
con un'assegnazione ad una variabile locale, in questo caso chiamato \verb|origine|. 

Tramite questo riferimenti è possibile accedere ai metodi, ed eventualmente allo stato dell'oggetto, tramite la notazione puntata, che segue la sintassi 
\verb|<riferimento-oggetto>.<metodo>(<parametri-attuali>);|:  
\begin{minted}{java}
origine.setX(0); // assegna lo stato alla variabile x
origine.setY(0); // assegna lo stato alla variabile y
\end{minted}
Questa notazione è molto efficace nell'esprimere la vicinanza tra i dati e le operazioni, per cui viene ampiamente usata in tutti i linguaggi OO (Orientati agli Oggetti). 
Ogni nuovo oggetto creato viene salvato in memoria, ed è possibile creare nuovi oggetti fino all'esaurimento della memoria virtuale. 
Questa classe viene quindi contenuta in un file \verb|Punto.java|. 

%% ? Per compilare un codice bisogna rendere il compilatore in grado di poter leggere il codice ... ?

Una variabile contenente un riferimento ad un oggetto contiene una sequenza di caratteri che rappresenta l'indirizzo in memoria virtuale dell'oggetto, anche se rappresenta una 
semplificazione, è utile per analizzare il suo comportamento. Mentre per accedere ai valori contenuti, ed assegnare valori alle variabili interne all'oggetto, bisogna utilizzare 
metodi ``getter'' e ``setter'', descritti nella classe dell'oggetto. 

Utilizzando due riferimenti allo stesso oggetto per effettuare azioni sullo stesso oggetto si chiama effetto collaterale. 

Quando viene chiamato un metodo passando una variabile, viene passato per valore. 

In Java il riferimento nullo è un unicamente il letterale di tipo riferimento all'oggetto \verb|null|. Ogni riferimento ad oggetto dichiarato, senza essere inizializzato 
viene assegnato al riferimento nullo. Indica l'assenza di un riferimento reale ad un oggetto esistente, al contrario del riferimento nullo in C, non corrisponde all'intero $0$. 
Invocando un metodo su un riferimento ad un oggetto nullo si verifica un errore, a tempo di esecuzione, si indica in Java come eccezione: ``runtime-exception''; in questo caso: 
\verb|NullPointerException|. 

%% !! ppt.

\subsection{Gestione della Memoria}

In Java la creazione di oggetti, e quindi la loro allocazione in memoria, avviene tramite l'operatore \verb|new|. Questo operatore richiede la specifica della classe di cui si 
vuole creare una nuova istanza, e richiede uno dei costruttori della classe. 
Un costruttore è un metodo che presenta esattamente lo stesso nome della classe e non restituisce niente, e generalmente inizializza tutte le variabili di istanza della classe. 

\begin{minted}{java}
public class Punto{
    private int x;
    private int y;
    // costruttore "no-args" della classe "Punto"
    public Punto(){
        this.x = 0;
        this.y = 0;
    }
}
\end{minted}
Questo tipo di costruttori non prendono parametri, ma è possibile definire costruttori che accettano degli input. Ogni classe presenta sempre almeno un costruttore, e 
costruisce lo stato iniziale dell'oggetto, per cui è possibile inizializzare l'oggetto ad un certo stato fornito alla chiamata del costruttore:
\begin{minted}{java}
public class Punto{
    private int x;
    private int y;
    // costruttore della classe "Punto"
    public Punto(int x, int y){
        this.x = x;
        this.y = y;
    }
}
\end{minted}

Al termine dell'esecuzione del costruttore, l'oggetto non è stato completamente inizializzato %% ??


Poiché una classe può contenere più di un costruttore, questi vengono distinti in base agli argomenti che prendono. 
Se invece non viene dichiarato alcune costruttore all'interno di una classe, è possibile creare un nuovo oggetto usando la stessa terminologia, e tutti i campi numerici 
vengono inizializzati a zero, i riferimenti a null, e le stringhe alla stringa vuota. Per cui il compilatore crea automaticamente un compilatore "no-args" con le inizializzazioni 
di default, ma se viene dichiarato un costruttore con parametri all'interno di una classe, e si prova a creare un oggetto della classe usando un costruttore ``no-args'' il 
compilatore produce un errore. Quindi il compilatore genera un costruttore implicito solamente se non è dichiarato alcun costruttore dentro una classe. 

%% !! ppt. Stack e Heap cartacei

Poiché gli oggetti si passano come riferimenti, l'operatore \verb|==| confronta solamente i riferimenti all'oggetto, e bisogna utilizzare un altro metodo per poter 
confrontare i valori interni degli oggetti tra di loro.  
Per determinare l'equivalenza tra due oggetti distinti si usa il metodo \verb|equals()|, viene offerto sempre da tutte le classi, se invece non viene espresso esplicitamente 
ha la stessa semantica dell'operatore \verb|==| sui riferimenti. 
Poiché è un metodo interno ad una classe, in generale non è un metodo simmetrico: \verb|a.equals(b)| $\neq$ \verb|b.equals(a)|. 

La deallocazione della memoria avviene automaticamente da un metodo chiamato Garbage Collector, per cui non è necessario utilizzare esplicitamente una funzione \verb|free()| 
come in C. Se questo Garbage Collector si accorge che degli oggetti rimangono appesi, li libera dalla memoria. 

\subsection{Costruttori, Stringhe e Array}

%% overloading
L'overloading è una caratteristica di un linguaggio di programmazione, come Java, che permette di definire più procedure con lo stesso nome, ospitati all'interno della stessa 
classe. Questi metodi, diversi, si distinguono solamente per la lista dei parametri formali, per il numero, per il tipo di uno o più, oppure per il loro ordine. 

%% ex. di metodi sovraccaricati
\begin{minted}{java}
public class Sommatore{
    public int add(int a, int b){
        return a + b;
    }
    public int add(int a, int b, in c){
        return a + b + c;
    }        
    public double add(int a, double b){
        return a + b;
    }
    public double add(double a, int b){
        return a + b;
    }
}
\end{minted}

Nel linguaggio C, questo non è possibile. Se c'è una corrispondenza perfetta tra i parametri passati al metodo, non si riscontrano problemi, invece se non è presente 
una corrispondenza perfetta tra i tipi passati al metodo, viene eseguito un algoritmo di risoluzione, prima di fermare la compilazione che cerca di capire se è possibile 
effettuare la chiamata tramite semplici conversioni di tipo, oppure conversioni di tipo più ``conservative''. La promozioni di tipo può essere implicita alla chiamata di un 
metodo, per cui potrebbe essere complesso determinare quale metodo è stato chiamato. 

Alla chiamata di un metodo sovraccaricato è il compilatore a scegliere quale dei metodi da applicare, la scelta è definitiva e viene scritta nell'eseguibile \verb|.class|. 
Versioni sovraccariche di un metodo potrebbero differire anche per il metodo di ritorno, ma il tipo di ritorno non può essere usato per distinguere due metodi. 

L'overloading è presente anche sull'operatore \verb|+|, applicandolo su diversi dati, tramite somme algebriche di interi, numerali a virgola mobile; inoltre viene usato per 
concatenare le stringhe. In C++ ed in Scala è possibile sovraccaricare gli altri operatori, mentre non è possibile in Java, dove l'unico operatore sovraccarico è \verb|+|, 
nella maniera descritta. 

Costruttori definiti ripetendo codice, vengono sconsigliati, anche per i metodi sovraccaricati, per cui si tende ad eleggere un costruttore al ruolo di costruttore ``primario'', 
il più generico possibile. In seguito gli altri costruttori vengono definiti in base a questo costruttore, invocando questo costruttore, con il metodo \verb|this()|, 
specificando i parametri che si vuole modificare rispetto al costruttore primario. 

%% ex della classe rettangolo 

\begin{minted}{java}
public class Rettangolo{
    private int altezza;
    private int base;
    private Punto vertice;

    public Rettangolo(Punto v, int b, int h){
        this.vertice = v;
        this.base = b;
        this.altezza = h;
    }

    public Rettangolo(int b, int h){
        this(new Punto(0,0), b, h);
    }

    public Rettangolo(){
        this(new Punto(0,0), 0, 0);
    }
}   
\end{minted}



%% la classe string
In Java esiste la classe \verb|String| per poter rappresentare sequenze di caratteri immutabili, rappresenta un riferimento ad un oggetto istanza della classe \verb|String|. 
Per rendere il linguaggio più semplice questa classe possiede dei letterali appositi, ed è l'unico oggetto che può essere creato senza una \verb|new| esplicita. L'operatore 
\verb|==| su due stringhe verifica l'uguaglianza dei riferimenti, per cui per valutare l'equivalenza di contenuto tra due stringhe bisogna utilizzare il metodo \verb|equals()|. 

Poiché rappresenta una sequenza di caratteri immutabili, per modificare o aggiornare una stringa è necessario creare una nuova stringa, altrimenti è possibile concatenare due 
stringhe utilizzando l'operatore \verb|+|, che rappresenta un nuovo oggetto stringa, sovrascritto sulla variabile utilizzata per memorizzarla. 
Per ottenere la lunghezza di una stringa si utilizza il metodo \verb|.length()|, per ottenere il carattere ad una determinata posizione si utilizza il carattere \verb|.charAt(int i)|. 
Il metodo \verb|.indexOf(char a)| fornisce la posizione di un carattere o il primo carattere di una sottostringa, per cui è un metodo sovraccarico, altrimenti restituisce $-1$. 
Per aggiornare una stringa si utilizza il metodo \verb|.replace(Stringa s1, Stringa s2)|, che restituisce un nuovo oggetto dove è stata la sottostringa \verb|s1| con la 
sottostringa \verb|s2|.  



Se viene eseguito il metodo \verb|println()| su un riferimento stampa il valore del riferimento, non il riferimento stesso. Definendo il metodo \verb|toString()| è possibile 
specificare cosa si vuole visualizzato quando viene chiamato il metodo di stampa a schermo. Un oggetto che possiede questo metodo può essere concatenato con una stringa, 
questo concatena la stringa con la string restituita dal metodo \verb|toString()|. 
Se non viene definito questo metodo, il suo comportamento standard implicito stampa l'indirizzo di memoria dell'oggetto, come per il metodo \verb|equals()|. 



Un array è una struttura dati che memorizza dati omogenei, per dichiarare un array si inserisce accanto al tipo oppure al nome della variabile \verb|[]|, si preferisce la prima 
per leggibilità. Sono oggetti per cui devono essere dichiarati con il metodo \verb|new|. Le operazioni di accesso e modifica di un array sono le medesime del linguaggio C, 
per ottenere la lunghezza di un array si utilizza il metodo \verb|.length()|. Da Java 5 è stato introdotto un modo per poter iterare su un array, senza gestire 
esplicitamente l'indice di iterazione \verb|for(<tipo> <nome-variabile> : <nome-array>)|, in questo modo la variabile \verb|<nome-variabile>| itera su ogni elemento dell'array 
\verb|<nome-array>| dello stesso tipo \verb|<tipo>|. 

\clearpage

\section{Qualità del Codice}

\subsection{Librerie}

Come ogni linguaggio di programmazione, relativamente, recente, Java contiene numerose librerie specializzate per risolvere diversi problemi. Per cui è necessario utilizzare in 
maniera efficace ed efficiente queste librerie, composte da classi. Bisogna quindi conoscere per nome le classi più importanti di queste librerie, e sapere come cercare ed usare 
altre classi. 
Non è necessario conoscere l'implementazione di una classe, ma solo l'interfaccia o API fornito. Queste informazioni vengono fornite tramite la documentazione associata ad ogni 
libreria, in formato HTML \verb|javadoc|, accessibili tramite un browser, oppure integrate all'interno dell'IDE. 
Per ogni classe viene fornito il suo nome, una descrizione generale della classe, dello scopo dei suoi costruttori e metodi, e valori di ritorno per i suoi costruttori e metodi. 
Non sono presenti invece tutti i campi e metodi privati, poiché non sono accessibili dall'esterno, né il suo codice, non necessario per il suo utilizzo. 

In caso si voglia creare una nuova classe, bisogna documentarla allo stesso modo delle classi fornite dalle librerie, per permettere il loro utilizzo, senza conoscere in 
dettaglio la sua implementazione. La documentazione di una classe dovrebbe includere il nome, un commento che descrive la caratteristiche generali e lo scopo della classe, la versione, 
il nome dei metodi e dei costruttori, una descrizione del suo scopo, i parametri di ritorno e passati al costruttore o metodo. 
La documentazione può essere generata automaticamente tramite un'utility chiamata \verb|javadoc|, contenuta in Eclipse, i marcatori utilizzati per definire i comandi si trovano 
esclusivamente all'interno  di un commento generato del genere \verb|/** <comandi> */|:
\begin{minted}{java}
/**
 * Nome-classe: breve descrizione della classe ed il suo scopo
 * 
 * @author      autore della classe
 * @see         riferimento ad altre classi
 * @version     versione corrente della classe
 */  
\end{minted}

I commenti per un metodo o un costruttore vengono inseriti prima subito prima del metodo o costruttore stesso. 
\begin{minted}{java}
/**
 * Commento che descrive scopo e caratteristiche del metodo o costruttore
 * 
 * @param nome-parametro    breve descrizione
 * @return                  valore di ritorno
 */
\end{minted}

La classi vengono raggruppate in \verb|package|, per mantenere insieme classi concettualmente e logicamente correlate. Permette di creare spazi di nomi per evitare conflitti, 
in oltre permette di definire domini di protezione. 
Una classe può accedere a tutte le classi presenti nello stesso \verb|package|, altrimenti per accedere a classi pubbliche di un altro pacchetto si può utilizzare il nome completo, 
anteponendo il nome del pacchetto, oppure importando la classe ed utilizzando direttamente il nome della classe. Le classi devono dichiarare la propria appartenenza ad un 
pacchetto tramite la dichiarazione \verb|package <nome-package>|, all'inizio del file. Una classe può appartenere a più di un pacchetto. Ogni pacchetto deve avere un nome univoco, 
di solito il nome comprende il nome del dominio Internet dell'organizzazione in ordine inverso. Una classe può essere usata al di fuori del pacchetto solamente se viene dichiarata 
pubblica, tramite un modificatore d'accesso \verb|pubblic|, se viene omesso questo modificatore, è accessibile solamente alle classi interne allo stesso pacchetto.

Il nome di un pacchetto possiede una struttura gerarchica, questa struttura deve avere una corrispondenza diretta nel file system. 

\subsection{Coesione e Accoppiamento}

Il software evolve continuamente, viene modificato in continuazione, per estensioni, correzioni, mantenimento, etc., e se il costo di questa evoluzione è troppo elevato, il 
software viene gettato e viene implementato da zero. Questa evoluzione viene effettuata in tempi diversi da molte persone. 
Se il codice è di cattiva qualità, la sua manutenzione ha un costo relativamente alto. La qualità del codice dipende da due fattori importanti, la coesione e l'accoppiamento. 
L'accoppiamento tra due o più unità di un programma rappresenta l'impossibilità di modificare una sola unità senza l'eventuale modifica di altre unità accoppiate ad essa. 
Per cui si vuole evitare l'accoppiamento, per limitare il più possibile la modifica di ulteriori unità, poiché aumenterebbe esponenzialmente i costi di una singola modifica. 
Un sintomo del forte accoppiamento è la duplicazione del codice all'interno di un progetto. Un basso accoppiamento permette di capire il codice di una classe senza leggere 
i dettagli di tutte le altri a lei accoppiata, e quindi anche la modifica, senza modificare ulteriori unità. 

La coesione si riferisce al numero e all'eterogeneità dei compiti di cui una singola classe è responsabile. Se ogni unità è responsabile di un singolo compito, allora possiede 
un'elevata coesione. La coesione si applica alle classi, ai metodi ed anche ai pacchetti. Un'alta coesione permette di definire moduli aventi uno scopo preciso ed un compito ben 
definito. Questo favorisce la comprensione dei compiti di una classe, l'utilizzo di nomi appropriati, ed il riuso di classi e di metodi, rendendo la manutenzione, meno costosa. 
Alta coesione e basso accoppiamento rappresentano due facce della stessa medaglia. %% wording


Per cui quando viene progettato il codice è utile pensare a quali cambiamenti futuri potranno essere richiesti, e come verrà usata la classe. 
%% ppt.


La manutenzione del codice spesso richiede l'aggiunta di nuovo codice, questo eventualmente degrada il codice nella sua totalità, aumentando l'accoppiamento e diminuendo la 
coesione, portando ad un'eventuale riorganizzazione del codice. Questa operazione viene chiamata ``refactoring'', senza modificare la funzionalità del codice. Per mantenere 
la correttezza del codice viene associato a dei test che stabiliscono i criteri di correttezza del codice, usati per valutare la correttezza di implementazioni interne diverse. 


In generale un metodo è troppo lungo se è responsabile di più di un compito logico, mentre una classe è troppo lunga se comprende più di un concetto. 


\subsection{Tecniche di Testing}

In generale i primi errori in un programma sono errori di sintassi, indicati dal compilatore in modo preciso e completo, anche se le specifiche di un messaggio di errore 
dipendono dal compilatore utilizzato, successivamente possono presentarsi errori logici o ``bug'', su cui il compilatore non è in grado di fornire indicazioni. 
Alcuni errori logici non si presentano immediatamente, per la complessità del software, altri possono essere individuati a tempo di esecuzione, e la macchina virtuale 
fornisce informazioni precise sulla natura dell'errore, la maggior parte delle volte si tratta di un \verb|NullPointerException|. 

Un programma rappresenta una singola descrizione statica associata a molteplici possibili esecuzioni dinamiche. Il compilatore è in grado di individuare errori a livello statico, 
a tempo di scrittura o compilazione, e non è in grado di individuare possibili errori dovuti all'esecuzione e l'evoluzione del programma. 
I bug sono quindi errori che si presentano durante l'evoluzione dinamica del programma, che il compilatore non è in grado di prevedere. Il costo delle operazioni di debugging, 
rappresenta il costo principale di ogni moderno progetto di software, ed è interamente a carico del programmatore. 

La correzione di un bug dipende da due grandezze, la dimensione del contesto, ovvero il numero di linee di codice in cui il bug si può annidare. Inoltre dipende dal tempo 
necessario al bug per manifestarsi, rappresenta una misura temporale tra la causa del bug ed il rilevamento dei suoi effetti. 


Per rilevare i malfunzionamenti di un software vengono utilizzati delle tipologie di test, in tre fasi sequenziali:
\begin{itemize}
    \item Mettere il sistema in uno stato iniziale noto;
    \item Iniziare a sollecitare il sistema;
    \item Controllare lo stato del sistema e confrontarlo con lo stato atteso. 
\end{itemize}

Se i test sono progettati e mantenuti, allora permettono di individuare tempestivamente i bug all'interno del software, restringendo le cause del singolo bug. Se il test ha 
successo si ha una garanzia sul comportamento dinamico del codice, altrimenti fornisce informazioni sulla presenza di un bug. 
Ad ogni modifica è necessario ripetere un'operazione di testing, per identificare eventuali bug, inseriti durante la manutenzione del codice, in questo modo è possibile 
ricercare localmente l'errore, in maniera molto economica. Inoltre utilizzando questo processo è possibile prevenire la regressione, localizzando l'errore nel codice appena 
aggiunto al programma. 


Per automatizzare questo processo si realizza, accanto al codice di produzione, il codice di test. Una possibile soluzione inizializza tutti i casi da testare sul codice 
di produzione, inserendo, ad ogni chiamata, il valore atteso confrontandolo con il valore ottenuto. 
I test devono essere automatici, efficienti ed isolati, per mantenere la località degli errori, inoltre devono essere separati dal codice applicativo ed eseguibili e 
verificabili separatamente. 
Esistono vari strumenti per assistere il programmatore nel testing, in particolare l'unit-testing, in Java il più note ed utilizzato framework è ``JUnit'', integrato 
all'interno di Eclipse. 
Per ogni classe creata si crea una classe parallela dedicata al testing, contenente una batteria di test, ognuno un metodo differente per testare diversi casi, preceduto 
dall'annotazione \verb|@Test|, contenente un'asserzione sul risultato aspettato tramite il metodo \verb|assertEquals()|. 

Tutte le classi di test hanno la stessa struttura, si inseriscono nello stesso pacchetto della classe che devono testare, e vengono chiamate rispettando la convenzione 
\verb|test<nome-classe>|, anche se non è più necessario utilizzare questa notazione, semplicemente tramite l'annotazione \verb|@Test|, anche se viene favorito il suo 
utilizzo. I risultati attesti sono documentati tramite asserzioni esplicite, non mediante stampe, se l'asserzione è vera il test è andato a buon fine, altrimenti è 
presente un errore nella sezione di codice testato. Sono possibili altre asserzioni fornite da JUnit, sovraccariche e facilmente intercambiabili, per cui è sempre 
favorito utilizzare la versione più pertinente. 

Una variante di \verb|assertEquals()|, prevede una stringa, da stampare per fornire informazioni sul risultato del test.  

%% TODO uso + implementazione di JUnit su Eclipse

Per facilitare la scrittura di tutti i metodi di test, è comodo utilizzare degli oggetti predisposti per l'utilizzo da parte di tutti i test-case operazione chiamata 
Fixture. JUnit permette di confinare in un unico metodo \verb|setUp()| la creazione di tutti gli stati noti a priori tramite l'annotazione \verb|@BeforeEach|. 
In questo modo i test diventano estremamente semplici. 
La lunghezza ottimale di un singolo test-case consiste di una singola riga di codice, rappresentata dall'asserzione. Per favorire la semplicità dei test si può 
fattorizzare il codice di creazione della fixture. 
L'uso di test minimali favorisce la leggibilità e la ricerca di errori, ed è sempre conveniente partire da test minimali per individuare i bug minori, per poi 
aumentare la complessità dei test ed individuare bug sempre più complessi. 

In generale è buona norma scrivere i test prima ancora di aver scritto il programma, una tecnica chiamata ``test-driven-develompent''. 

\clearpage

\section{Interfacce e Polimorfismo}

\subsection{Upcasting e Downcasting}

In Java i riferimenti sono tipati, ovvero specificano il tipo dell'oggetto referenziato, quindi attraverso un riferimento possono essere eseguiti tutti i metodi offerti dalla 
classe dell'oggetto. In generale nei linguaggi orientati agli oggetti esiste più di un modo per poter creare un tipo, in Java oltre al costrutto \verb|class| esiste il 
costrutto \verb|interface|, utilizzato per creare un nuovo tipo. 
Questo costrutto specifica un tipo in termini dei metodi che può offrire, specificando solamente la segnatura ed il tipo di ritorno, per cui non è presente alcun dettaglio 
implementativo, come le variabili, costruttori ed il corpo dei metodi. Una classe può implementare una o più \verb|interface| tramite il costrutto \verb|implements <nome-interface>|, 
e deve contenere tutti i metodi interni contenuti dall'\verb|interface|, ma può contenete non solo questi metodi. 

Una classe che implementa un'\verb|interface| è un suo sottotipo, mentre l'\verb|interface| è un supertipo, una sua generalizzazione. 
In Java vale il principio di sostituzione, di Liskov, il quale afferma che un sottotipo può essere usato al posto di un suo supertipo. 


Per il principio di sostituzione, un riferimento ad un sottotipo può essere assegnato ad un riferimento ad un suo supertipo. La promozione di un tipo ad un suo supertipo viene 
chiamato ``upcasting''. 
Il collegamento tra l'assegnatura e l'implementazione di un \verb|interface| non è nota a tempo di compilazione, per cui si chiama ``late binding''. 
Quindi si può differenziare tra il tipo statico, definito a tempo di compilazione, ed il tipo dinamico, ciò che viene utilizzato a tempo di esecuzione. Il compilatore 
permette di applicare solo i metodi del tipo statico, poiché essendo legato solamente a tempo di esecuzione, il compilatore non permette di utilizzare i metodi del tipo 
dinamico. 

L'overloading dei metodi avviene a tempo di compilazione, quindi staticamente, ma in questo modo non tiene conto dei metodi dei tipi dinamici. 

%% TODO ppt. 8/4%
% 12/4: verifica polimorfismo

Le invocazioni polimorfe si risolvono a tempo di esecuzione sulla base del tipo dinamico da parte della macchina virtuale, l'unica componente in grado di avere informazioni 
sul tipo dinamico. Mentre l'overloading di un metodo viene risolto a tempo di compilazione dal compilatore, sulla base del tipo statico dei parametri passati al metodo. 

\subsection{Estensioni di Interface e Classi}

Date due interfacce diverse, aventi gli stessi metodi, anche se concettualmente sembrerebbero uguali, non è possibile referenziare un oggetto che implementa una delle due interfacce 
con un oggetto che implementa l'altra, poiché il linguaggio è staticamente tipato. Per risolvere questo problema è viene fornita la parola chiave \verb|extends <nome-interfaccia>| per 
specificare che un'interfaccia è sottotipo di un'altra interfaccia, aggiungendo dei metodi non presenti nella prima interfaccia. A tempo di compilazione è quindi possibile 
referenziare due oggetti che implementano queste due interfacce diverse, applicando il principio di sostituzione. 

Attraverso quest'estensione è possibile quindi definire nuovi tipi a partire da tipi già esistenti. 


Oltre alle interfacce, è possibile estendere le classi, ma questo è un meccanismo più complesso, poiché entrambe le classi, supertipo e sottotipo, contengono metodi con un corpo, 
a differenza delle interfacce. La classe di partenza viene chiamata superclasse, o classe base o genitore. La classe definita per estensione da questa, viene chiamata classe estesa, 
derivata, sottoclasse o figlia. 
La classe derivata ``eredita'' tutto quello che offre la classe di partenza. Tramite l'annotazione \verb|@Override| è possibile modificare un metodo con corpo della superclasse, 
all'interno della classe derivata. 
Per indicare la definizione di una classe come estensione di una classe si usa la stessa parola chiave \verb|extends <nome-classe>|. Queste classi derivata possono dichiarare 
metodi e membri aggiuntivi, e modificare tramite annotazioni metodi preesistenti. La classe base viene considerata un supertipo della classe estesa. Poiché rappresenta una classe 
esterna, non può accedere a parametri privati, interni alla classe base. Per accedere a queste variabili è possibile utilizzare i metodi getter e setter offerti dalla classe b
base, applicati sulla classe derivata. 
Come per le interfacce, non è possibile invocare i metodi privati, di una classe estesa, applicato sulla sua superclasse. 
Il meccanismo di modifica un metodo presente nella classe base si chiama ``overriding'', tramite annotazioni, come descritto precedentemente. Ma bisogna considerare che non 
può accedere a parti private della superclasse. 

%% TODO 15/4: ppt. 

Per creare un'istanza della sottoclasse è necessario invocare la superclasse corrispondente. Il costruttore di una classe estesa deve quindi inizializzare i propri campi, ed i 
campi propri della superclasse, ma un metodo di una sottoclasse non può accedere a parti private della sua superclasse. In java questo viene risolto delegando questo metodo, 
effettuando una chiamata al costruttore della sua superclasse, per inizializzare le sue parti private. Ciò si effettua tramite la parola chiave \verb|super(<parametri>)|, 
passando i parametri formali che diventano parametri attuali del costruttore della sua superclasse. Questa invocazione deve essere l'unica e sola nel costruttore, e deve 
essere presente alla prima riga del costruttore della sottoclasse. Questo poiché è necessario creare la superclasse prima di poter creare la sua sottoclasse, si è quindi vincolati 
quest'ordine sintattico. 

Se la superclasse non ha costruttori, il compilatore ne crea uno automaticamente senza argomenti, quindi è possibile che aggiungendo un costruttore della superclasse, si 
genera un errore di compilazione su un costruttore della sua sottoclasse, poiché utilizza il costruttore creato automaticamente dal compilatore ``no-arg''. 

\subsection{Classe Object e Overriding}

Ogni oggetto Java direttamente o indirettamente deve essere un sottotipo della classe \verb|Object|, chiamato anche la radice della gerarchia dei tipi Java.
Tutte le classi estendono automaticamente questa classe predefinita, contiene quindi alcuni metodi disponibili ad ogni classe. 
Alcuni di questi metodi sono \verb|toString()| e \verb|equals(Object o)|, tutte le classi ereditano questa implementazione, oltre la loro segnatura, possono quindi ridefinire 
l'implementazione, rispettando la segnatura. Il metodo \verb|toString()| di default stampa l'indirizzo di memoria dell'oggetto su cui viene applicato il metodo. 
Quasi sempre quindi conviene ridefinire questi metodi sulla base dello specificità della classe definita. 
Essendo tutte le classi estensioni di questa classe predefinita, ogni costruttore contiene implicitamente una chiamata al costruttore alla classe \verb|Object|. 
Generalmente quando si ridefinisce un metodo fornito dalla classe \verb|Object| bisogna effettuare un downcast al tipo della classe necessaria nel metodo. 

%% TODO 17/4 estensione studio di caso

La parola chiave \verb|super| è in grado di effettuare una chiamata ad un metodo presente nella superclasse, all'interno di una classe avente lo stesso nome nella 
sottoclasse. Impedendo di avere un invocazione ricorsiva dello stesso metodo fino all'esaurimento dello stack. 


In Java esiste un altro modificatore di accesso di livello intermedio tra \verb|public| e \verb|private|, chiamato \verb|protected|, permette ad una superclasse aventi 
membri protetti di essere visibili a tutte le sue sottoclassi, indipendentemente dai pacchetti di appartenenza. 
In generale questo modificatore di accesso viene evitato per applicazioni semplici, come lo studio di caso trattato in questo corso, poiché rappresenta uno studio 
introduttivo. Infatti contraddice implicitamente il principio dell'``information hiding''. 
L'utilizzo più opportuno è nella progettazione di framework, librerie che permettono l'estensione da parte degli utilizzatori. 

Il livello di visibilità ottenuto senza inserire un modificatore di accesso viene chiamato anche \verb|package-private|. Questo livello di visibilità è il meno permissivo 
dopo il modificatore \verb|private|. 

%% TODO matrice dei livelli di accesso

In ordine di visibilità si ha quindi in ordine decrescente \verb|public|, \verb|protected|, \verb|package-private| e \verb|private|. 

%% TODO metodi sovrascritti

Sovrascrivendo un metodo è possibile modificare la sua visibilità, rendendolo più visibile, ma non è possibile diminuire la visibilità di uno metodo da una sottoclasse. 


Il metodo di una sottoclasse può modificare un metodo, ereditato dalla superclasse, di identica segnatura ma di tipo restituito più generale. Questa caratteristica si chiama 
covarianza. 
Mentre si può effettuare un procedimento analogo con parametri polimorfi, dove bisogna posizionare nel supertipo il parametro più specifico mentre nel sottotipo il parametro 
più generico. Questa caratteristica si chiama controvarianza. 
Per cui il parametro passato è controvariante, mentre il parametro restituito è covariante; scendendo nella gerarchia dei tipi, i parametro passati devono essere sempre 
più generali, mentre i parametri restituiti sempre più specifici. 

Poiché in Java due metodi aventi lo stesso tipo rappresentano un sovraccarico, quindi questo tipo di overload viene risolto a tempo statico, ma secondo il paradigma orientato 
agli oggetti questo rappresenta una sovrascrittura, nonostante siano due metodi che prendono due parametri distinti polimorfi. Quindi utilizzando la parola chiave \verb|@Override| 
viene generato un eccezione a tempo di compilazione. 

In Java la gerarchia delle classi ha una radice definita, la radice di Object è quindi \verb|java.lang.Object|, e rappresenta la radice predefinita della gerarchia delle classi. 
Al contrario di altri linguaggi di programmazione in Java, al contrario di C++, non è possibile avere una derivazione multipla tra classi, una class può estendere al più una 
sola altra classe, e può avere zero o più sottoclassi. 
%% TODO problema del fro/go/sauro
In Scala questo problema viene risolto creando un'ordine totale, tramite un processo di linearizzazione della gerarchia in base a quale tipo viene prima all'interno. 
Se un membro è definito in entrambi le classi base, la classe estesa da queste due mischia le due implementazioni. Quando le due classi non presentano membri che 
si sovrappongono tra di loro è possibile unire le due classi poiché non si presentano conflitti. 

In Java invece è possibile effettuare una derivazione multipla tra interfacce, la stessa classe \verb|String| implementa tre interfacce. Questo è possibile poiché non è 
possibile mischiare il corpo dei metodi forniti dalle interfacce, poiché vengono implementati all'interno della classe che le estende. 
La gerarchia dei tipi non è più lineare, poiché comprende anche le interfacce, che possono estendere zero o più interfacce, mentre le classi possono implementare zero o più 
interfacce. 
La gerarchia delle classi è lineare, mentre la gerarchia delle interfacce non è lineare. In Java 8 vennero introdotti i metodi default, che permettono una derivazione multipla 
ristretta, per poter estendere librerie senza perdere la retrocompatibilità. 


La scelta dei corretti tipi da definire in un programma è un delicato esercizio di modellazione. Nella creazione dei tipi occorre considerare anche gli aspetti dinamici. 


\end{document}